%
% Section 2.2
\hypersection{section2-2}{数値区切り文字}

数値区切り文字は、整数リテラルと浮動小数点数リテラルの数値をシングルクオート文字で区切ることができる機能だ。区切り桁は何桁でもよい。
\index{すうちくぎりもじ@数値区切り文字}

\begin{lstlisting}[language=C++]
int main()
{
    int x1 = 123'456'789 ;
    int x2 = 1'2'3'4'5'6'7'8'9 ; 
    int x3 = 1'2345'6789 ;
    int x4 = 1'23'456'789 ;

    double x5 = 3.14159'26535'89793 ;
}
\end{lstlisting}

大きな数値を扱うとき、ソースファイルに\lstinline!100000000!と\lstinline!1000000000!と書かれていた場合、どちらが大きいのか人間の目にはわかりにくい。人間が読んでわかりにくいコードは間違いの元だ。数値区切りを使うと、\lstinline!100'000'000と1'000'000'000!のように書くことができる。これはわかりやすい。

他には、1バイト単位で見やすいように区切ることもできる。

\begin{lstlisting}[language=C++]
int main()
{
    unsigned int x1 = 0xde'ad'be'ef ;
    unsigned int x2 = 0b11011110'10101101'10111110'11101111 ;
}
\end{lstlisting}

数値区切りはソースファイルを人間が読みやすくするための機能で、数値に影響を与えない。
