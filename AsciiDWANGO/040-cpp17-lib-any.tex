%
% Section 4.2
\hypersection{section4-2}{any : どんな型の値でも保持できるクラス}
\index{any@\texttt{any}}

%
% SubSection 4.2.1
\hypersubsection{section4-2-1}{使い方}

ヘッダーファイル~\lstinline!<any>!~で定義されている\lstinline!std::any!は、ほとんどどんな型の値でも保持できるクラスだ。
\index{<any>@\texttt{<any>}}\index{std::any@\texttt{std::any}}

\begin{lstlisting}[language=C++]
#include <any>

int main()
{
    std::any a ;

    a = 0 ; // int
    a = 1.0 ; // double
    a = "hello" ; // char const *

    std::vector<int> v ;
    a = v ; // std::vector<int>

    // 保持しているstd::vector<int>のコピー
    auto value = std::any_cast< std::vector<int> >( a ) ;
}
\end{lstlisting}

\lstinline!any!が保持できない型は、コピー構築できない型だ。

%
% SubSection 4.2.2
\hypersubsection{section4-2-2}{anyの構築と破棄}
\index{any@\texttt{any}!こうちく@構築}\index{any@\texttt{any}!はき@破棄}

クラス\lstinline!any!はテンプレートではない。そのため宣言は単純だ。

\begin{lstlisting}[language=C++]
int main()
{
    // 値を保持しない
    std::any a ;
    // int型の値を保持する
    std::any b( 0 ) ;
    // double型の値を保持する
    std::any c( 0.0 ) ;
}
\end{lstlisting}

\lstinline!any!が保持する型を事前に指定する必要はない。

クラス\lstinline!any!を破棄すると、そのとき保持していた値が適切に破棄される。

%
% SubSection 4.2.3
\hypersubsection{section4-2-3}{in\texttt{\_}place\texttt{\_}typeコンストラクター}
\index{emplace@\texttt{emplace}}\index{any@\texttt{any}!emplace@\texttt{emplace}}\index{std::in\_place\_type<T>@\texttt{std::in\_place\_type<T>}}\index{any@\texttt{any}!std::in\_place\_type<T>@\texttt{std::in\_place\_type<T>}}

\lstinline!any!のコンストラクターで\lstinline!emplace!をするために\lstinline!in_place_type!が使える。

\begin{lstlisting}[language=C++]
struct X
{
    X( int, int ) { }
} ;

int main()
{
    // 型XをX(1, 2)で構築した結果の値を保持する
    std::any a( std::in_place_type<X>, 1, 2 ) ;
}
\end{lstlisting}

%
% SubSection 4.2.4
\hypersubsection{section4-2-4}{anyへの代入}
\index{any@\texttt{any}!だいにゆう@代入}

\lstinline!any!への代入も普通のプログラマーの期待どおりの動きをする。

\begin{lstlisting}[language=C++]
int main()
{
    std::any a ;
    std::any b ;

    // aはint型の値42を保持する
    a = 42 ;
    // bはint型の値42を保持する
    b = a ;
    
}
\end{lstlisting}

%
% SubSection 4.2.5
\hypersubsection{section4-2-5}{anyのメンバー関数}

%
% SubsubSection 4.2.5.1
\hypersubsubsection{section4-2-5-1}{emplace}
\index{emplace@\texttt{emplace}}\index{any@\texttt{any}!emplace@\texttt{emplace}}

\bgroup
\begin{lstlisting}[language=C++]
template <class T, class... Args>
decay_t<T>& emplace(Args&&... args);
\end{lstlisting}
\egroup

\lstinline!any!は\lstinline!emplace!メンバー関数をサポートしている。

\begin{lstlisting}[language=C++]
struct X
{
    X( int, int ) { }
} ;

int main()
{
    std::any a ;

    // 型XをX(1, 2)で構築した結果の値を保持する
    a.emplace<X>( 1, 2 ) ;
}
\end{lstlisting}

%
% SubsubSection 4.2.5.2
\vskip 1zw
\hypersubsubsection{section4-2-5-2}{reset : 値の破棄}
\index{reset@\texttt{reset}}\index{any@\texttt{any}!reset@\texttt{reset}}

\bgroup
\begin{lstlisting}[language=C++]
void reset() noexcept ; 
\end{lstlisting}
\egroup

\lstinline!any!の\lstinline!reset!メンバー関数は、\lstinline!any!の保持してある値を破棄する。\lstinline!reset!を呼び出した後の\lstinline!any!は値を保持しない。

\begin{lstlisting}[language=C++]
int main()
{
    // aは値を保持しない
    std::any a ;
    // aはint型の値を保持する
    a = 0 ;

    // aは値を保持しない
    a.reset() ;
}
\end{lstlisting}

%
% SubsubSection 4.2.5.3
\vskip 1zw
\hypersubsubsection{section4-2-5-3}{swap : スワップ}
\index{swap@\texttt{swap}}\index{any@\texttt{any}!swap@\texttt{swap}}

\lstinline!any!は\lstinline!swap!メンバー関数をサポートしている。

\begin{lstlisting}[language=C++]
int main()
{
    std::any a(0) ;
    std::any b(0.0) ;

    // aはint型の値を保持
    // bはdouble型の値を保持

    a.swap(b) ;

    // aはdouble型の値を保持
    // bはint型の値を保持
}
\end{lstlisting}

%
% SubsubSection 4.2.5.4
\vskip 1zw
\hypersubsubsection{section4-2-5-4}{has\texttt{\_}value : 値を保持しているかどうか調べる}
\index{has\_value@\texttt{has\_value}}\index{any@\texttt{any}!has\_value@\texttt{has\_value}}

\bgroup
\begin{lstlisting}[language=C++]
bool has_value() const noexcept;
\end{lstlisting}
\egroup

\lstinline!any!の\lstinline!has_value!メンバー関数は\lstinline!any!が値を保持しているかどうかを調べる。値を保持しているならば\lstinline!true!を、保持していないならば\lstinline!false!を返す。

\begin{lstlisting}[language=C++]
int main()
{
    std::any a ;

    // false
    bool b1 = a.has_value() ;

    a = 0 ;
    // true
    bool b2 = a.has_value() ;

    a.reset() ;
    // false
    bool b3 = a.has_value() ;
}
\end{lstlisting}

%
% SubsubSection 4.2.5.5
\vskip 1zw
\hypersubsubsection{section4-2-5-5}{type : 保持している型のtype\texttt{\_}infoを得る}
\index{type@\texttt{type}}\index{any@\texttt{any}!type@\texttt{type}}

\bgroup
\begin{lstlisting}[language=C++]
const type_info& type() const noexcept;
\end{lstlisting}
\egroup

\lstinline!type!メンバー関数は、保持している型\lstinline!T!の\lstinline!typeid(T)!を返す。値を保持していない場合、\lstinline!typeid(void)!を返す。

\begin{lstlisting}[language=C++]
int main()
{
    std::any a ;

    // typeid(void)
    auto & t1 = a.type() ;

    a = 0 ;
    // typeid(int)
    auto & t2 = a.type() ;

    a = 0.0 ;
    // typeid(double)
    auto & t3 = a.type() ;
}
\end{lstlisting}

%
% SubSection 4.2.6
\hypersubsection{section4-2-6}{anyのフリー関数}

%
% SubsubSection 4.2.6.1
\hypersubsubsection{section4-2-6-1}{make\texttt{\_}any\texttt{<}T\texttt{>} : T型のanyを作る}
\index{make\_any<T>@\texttt{make\_any<T>}}\index{any@\texttt{any}!make\_any<T>@\texttt{make\_any<T>}}

\bgroup
\begin{lstlisting}[language=C++]
template <class T, class... Args>
any make_any(Args&& ...args);

template <class T, class U, class... Args>
any make_any(initializer_list<U> il, Args&& ...args);
\end{lstlisting}
\egroup

\lstinline!make_any<T>( args... )!は\lstinline!T!型をコンストラクター実引数\lstinline!args...!で構築した値を保持する\lstinline!any!を返す。

\begin{lstlisting}[language=C++]
struct X
{
    X( int, int ) { }
} ;

int main()
{
    // int型の値を保持するany
    auto a = std::make_any<int>( 0 ) ;
    // double型の値を保持するany
    auto b = std::make_any<double>( 0.0 ) ;

    // X型の値を保持するany
    auto c = std::make_any<X>( 1, 2 ) ;
}
\end{lstlisting}

%
% SubsubSection 4.2.6.2
\hypersubsubsection{section4-2-6-2}{any\texttt{\_}cast : 保持している値の取り出し}
\index{any\_cast<T>@\texttt{any\_cast<T>}}\index{any@\texttt{any}!any\_cast<T>@\texttt{any\_cast<T>}}

\bgroup
\begin{lstlisting}[language=C++]
template<class T> T any_cast(const any& operand);
template<class T> T any_cast(any& operand);
template<class T> T any_cast(any&& operand);
\end{lstlisting}
\egroup

\lstinline!any_cast<T>(operand)!は\lstinline!operand!が保持している値を返す。

\begin{lstlisting}[language=C++]
int main()
{
    std::any a(0) ;

    int value = std::any_cast<int>(a) ;
}
\end{lstlisting}

\lstinline!any_cast<T>!~で指定した\lstinline!T!型が、\lstinline!any!が保持している型ではない場合、\lstinline!std::bad_any_cast!が\lstinline!throw!される。

\begin{lstlisting}[language=C++]
int main()
{
    try {
        std::any a ;
        std::any_cast<int>(a) ;
    } catch( std::bad_any_cast e )
    {
        // 型を保持していなかった
    }

}
\end{lstlisting}

\begin{lstlisting}[language=C++]
template<class T>
const T* any_cast(const any* operand) noexcept;
template<class T>
T* any_cast(any* operand) noexcept;
\end{lstlisting}

\lstinline!any_cast<T>!~に\lstinline!any!へのポインターを渡すと、\lstinline!T!へのポインター型が返される。\lstinline!any!が\lstinline!T!型を保持している場合は\lstinline!T!型を参照するポインターが返る。保持していない場合は、\lstinline!nullptr!が返る。

\begin{lstlisting}[language=C++]
int main()
{
    std::any a(42) ;

    // int型の値を参照するポインター
    int * p1 = std::any_cast<int>( &a ) ;

    // nullptr
    double * p2 = std::any_cast<double>( &a ) ;
}
\end{lstlisting}

