%
% Section 3.13
\hypersection{section3-13}{演算子のオペランドの評価順序の固定}

C++17では演算子のオペランドの評価順序が固定された。
\index{えんざんしのひようかじゆんじよ@演算子の評価順序}

以下の式は、\lstinline!a!, \lstinline!b!の順番に評価されることが規格上保証される。\lstinline!@=!~の~\lstinline!@!~には文法上許される任意の演算子が入る(\lstinline!+=!, \lstinline!-=!~など)。

\begin{lstlisting}[language=C++]
a.b
a->b
a->*b
a(b1,b2,b3)
b = a
b @= a
a[b]
a << b
a >> b
\end{lstlisting}

つまり、
\begin{lstlisting}[language=C++]
int* f() ;
int g() ;

int main()
{
   f()[g()] ; 
}
\end{lstlisting}
と書いた場合、関数\lstinline!f!がまず先に呼び出されて、次に関数\lstinline!g!が呼び出されることが保証される。

関数呼び出しの実引数のオペランド\lstinline!b1!, \lstinline!b2!,
\lstinline!b3!の評価順序は未規定のままだ。

これにより、既存の未定義の動作となっていたコードの挙動が定まる。
