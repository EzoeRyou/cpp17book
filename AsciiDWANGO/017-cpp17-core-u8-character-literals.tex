%
% Section 3.3
\hypersection{section3-3}{UTF-8文字リテラル}

C++17ではUTF-8文字リテラルが追加された。
\index{UTF-8もじりてらる@UTF-8文字リテラル}

\begin{lstlisting}[language=C++]
char c = u8'a' ;
\end{lstlisting}

UTF-8文字リテラルは文字リテラルにプレフィクス\lstinline!u8!を付ける。UTF-8文字リテラルはUTF-8のコード単位1つで表現できる文字を扱うことができる。UCSの規格としては、C0制御文字と基本ラテン文字Unicodeブロックが該当する。UTF-8文字リテラルに書かれた文字が複数のUTF-8コード単位を必要とする場合はエラーとなる。
\index{u8@\texttt{u8}}

\begin{lstlisting}[language=C++]
// エラー
// U+3042はUTF-8は0xE3, 0x81, 0x82という3つのコード単位で表現する必要が
// あるため
u8'あ' ;
\end{lstlisting}

機能テストマクロはない。
