%
% Section 2.1
\hypersection{section2-1}{二進数リテラル}

二進数リテラルは整数リテラルを二進数で記述する機能だ。整数リテラルのプレフィクスに\lstinline!0B!もしくは\lstinline!0b!を書くと、二進数リテラルになる。整数を表現する文字は0と1しか使えない。
\index{にしんすうりてらる@二進数リテラル}\index{\texttt{0B}}\index{\texttt{0b}}

\begin{lstlisting}[language=C++]
int main()
{
    int x1 = 0b0 ; // 0
    int x2 = 0b1 ; // 1
    int x3 = 0b10 ; // 2
    int x4 = 0b11001100 ; // 204
}
\end{lstlisting}

二進数リテラルは浮動小数点数リテラルには使えない。

機能テストマクロは~\lstinline!__cpp_binary_literals!, 値は201304。
\index{\_\_cpp\_binary\_literals@\texttt{\_\_cpp\_binary\_literals}}
