%
% Section 9.18
\hypersection{section9-18}{mapとunordered\texttt{\_}mapの変更}

\lstinline!map!と\lstinline!unordered_map!に、\lstinline!try_emplace!と\lstinline!insert_or_assign!という2つのメンバー関数が入った。このメンバー関数は\lstinline!multi_map!と\lstinline!unordered_multi_map!には追加されていない。
\index{map@\texttt{map}}\index{unordered\_map@\texttt{unordered\_map}}\index{multi\_map@\texttt{multi\_map}}\index{unordered\_multi\_map@\texttt{unordered\_multi\_map}}

%
% SubSection 9.18.1
\hypersubsection{section9-18-1}{try\texttt{\_}emplace}
\index{try\_emplace@\texttt{try\_emplace}}

\bgroup
\begin{lstlisting}[language=C++]
template <class... Args>
pair<iterator, bool>
try_emplace(const key_type& k, Args&&... args);

template <class... Args>
iterator
try_emplace(
    const_iterator hint,
    const key_type& k, Args&&... args);
\end{lstlisting}
\egroup

従来の\lstinline!emplace!は、キーに対応する要素が存在しない場合、要素が\lstinline!args!から\lstinline!emplace!構築されて追加される。もし、キーに対応する要素が存在する場合、要素は追加されない。要素が追加されないとき、\lstinline!args!がムーブされるかどうかは実装定義である。
\index{emplace@\texttt{emplace}}

\begin{lstlisting}[language=C++]
int main()
{
    std::map< int, std::unique_ptr<int> > m ;

    // すでに要素が存在する
    m[0] = nullptr ;

    auto ptr = std::make_unique<int>(0) ;
    // emplaceは失敗する
    auto [iter, is_emplaced] = m.emplace( 0, std::move(ptr) ) ;

    // 結果は実装により異なる
    // ptrはムーブされているかもしれない
    bool b = ( ptr != nullptr ) ;
}
\end{lstlisting}

この場合、実際に\lstinline!map!に要素は追加されていないのに、\lstinline!ptr!はムーブされてしまうかもしれない。

このため、C++17では、要素が追加されなかった場合\lstinline!args!はムーブされないことが保証される\lstinline!try_emplace!が追加された。

\begin{lstlisting}[language=C++]
int main()
{
    std::map< int, std::unique_ptr<int> > m ;

    // すでに要素が存在する
    m[0] = nullptr ;

    auto ptr = std::make_unique<int>(0) ;
    // emplaceは失敗する
    auto [iter, is_emplaced] = m.emplace( 0, std::move(ptr) ) ;

    // trueであることが保証される
    // ptrはムーブされていない
    bool b = ( ptr != nullptr ) ;
}
\end{lstlisting}

%
% SubSection 9.18.2
\hypersubsection{section9-18-2}{insert\texttt{\_}or\texttt{\_}assign}
\index{insert\_or\_assign@\texttt{insert\_or\_assign}}

\bgroup
\begin{lstlisting}[language=C++]
template <class M>
pair<iterator, bool>
insert_or_assign(const key_type& k, M&& obj);

template <class M>
iterator
insert_or_assign(
    const_iterator hint,
    const key_type& k, M&& obj);
\end{lstlisting}
\egroup

\lstinline!insert_or_assign!は\lstinline!key!に連想された要素が存在する場合は要素を代入し、存在しない場合は要素を追加する。\lstinline!operator []!との違いは、要素が代入されたか追加されたかが、戻り値の\lstinline!pair!の\lstinline!bool!でわかるということだ。

\begin{lstlisting}[language=C++]
int main()
{
    std::map< int, int > m ;
    m[0] = 0 ;

    {
        // 代入
        // is_insertedはfalse
        auto [iter, is_inserted] = m.insert_or_assign( 0, 1 ) ;
    }

    {
        // 追加
        // is_insertedはtrue
        auto [iter, is_inserted] = m.insert_or_assign( 1, 1 ) ;
    }
}
\end{lstlisting}

