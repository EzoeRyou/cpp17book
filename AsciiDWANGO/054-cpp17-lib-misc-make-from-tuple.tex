%
% Section 9.8
\hypersection{section9-8}{make\texttt{\_}from\texttt{\_}tuple : tupleの要素を実引数にコンストラクターを呼び出す}

\lstinline!make_from_tuple!はヘッダーファイル~\lstinline!<tuple>!~で定義されている。
\index{make\_from\_tuple@\texttt{make\_from\_tuple}}\index{<tuple>@\texttt{<tuple>}}

\begin{lstlisting}[language=C++]
template <class T, class Tuple>
constexpr T make_from_tuple(Tuple&& t);
\end{lstlisting}

\lstinline!apply!は\lstinline!tuple!の要素を実引数に関数を呼び出すライブラリだが、\lstinline!make_from_tuple!は\lstinline!tuple!の要素を実引数にコンストラクターを呼び出すライブラリだ。
\index{tuple@\texttt{tuple}}

ある型\lstinline!T!と要素数\lstinline!N!の\lstinline!tuple t!に対して、\lstinline!make_from_tuple<T>(t)!は、\lstinline!T!型を\lstinline!T( get<0>(t), get<1>(t), ... , get<N-1>(t) )!のように構築して、構築した\lstinline!T!型のオブジェクトを返す。

\begin{lstlisting}[language=C++]
class X
{
    template < typename ... Types >
    T( Types ... ) { }
} ;

int main()
{
    // int, int, int
    std::tuple t1(1,2,3) ;

    // X(1,2,3)
    X x1 = std::make_from_tuple<X>( t1 ) 

    // int, double, const char *
    std::tuple t2( 123, 4.56, "hello" ) ;

    // X(123, 4.56, "hello")
    X x2 = std::make_from_tuple<X>( t2 ) ;
}
\end{lstlisting}

