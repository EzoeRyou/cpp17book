%
% Section 2.11
\hypersection{section2-11}{サイズ付き解放関数}
\index{さいずつきかいほうかんすう@サイズ付き解放関数}

C++14では\lstinline!operator delete!のオーバーロードに、解放すべきストレージのサイズを取得できるオーバーロードが追加された。
\index{operator delete@\texttt{operator delete}}\index{delete@\texttt{delete}}

\begin{lstlisting}[language=C++]
void operator delete    ( void *, std::size_t ) noexcept ;
void operator delete[]  ( void *, std::size_t ) noexcept ;
\end{lstlisting}

第二引数は\lstinline!std::size_t!型で、第一引数で指定されたポインターが指す解放すべきストレージのサイズが与えられる。
\index{std::size\_t@\texttt{std::size\_t}}

たとえば以下のように使える。

\begin{lstlisting}[language=C++]
void * operator new ( std::size_t size )
{
    void * ptr =  std::malloc( size ) ;

    if ( ptr == nullptr )
        throw std::bad_alloc() ;

    std::cout << "allocated storage of size: " << size << '\n' ;
    return ptr ;
}

void operator delete ( void * ptr, std::size_t size ) noexcept
{
    std::cout << "deallocated storage of size: " << size << '\n' ;
    std::free( ptr ) ;
}

int main()
{
    auto u1 = std::make_unique<int>(0) ;
    auto u2 = std::make_unique<double>(0.0) ;
}
\end{lstlisting}

機能テストマクロは~\lstinline!__cpp_sized_deallocation!, 値は201309。
\index{\_\_cpp\_sized\_deallocation@\texttt{\_\_cpp\_sized\_deallocation}}
