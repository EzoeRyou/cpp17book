%
% Section 2.4
\hypersection{section2-4}{通常の関数の戻り値の型推定}
\index{かんすうのもどりちのかたすいてい@関数の戻り値の型推定}

関数の戻り値の型として\lstinline!auto!を指定すると、戻り値の型を\lstinline!return!文から推定してくれる。
\index{auto@\texttt{auto}}

\begin{lstlisting}[language=C++]
// int ()
auto a(){ return 0 ; }
// double ()
auto b(){ return 0.0 ; }

// T(T)
template < typename T >
auto c(T t){ return t ; }
\end{lstlisting}

\lstinline!return!文の型が一致していないとエラーとなる。

\begin{lstlisting}[language=C++]
auto f()
{
    return 0 ; // エラー、一致していない
    return 0.0 ; // エラー、一致していない
}
\end{lstlisting}

すでに型が決定できる\lstinline!return!文が存在する場合、関数の戻り値の型を参照するコードも書ける。

\begin{lstlisting}[language=C++]
auto a()
{
    &a ; // エラー、aの戻り値の型が決定していない
    return 0 ;
}

auto b()
{
    return 0 ;
    &b ; // OK、戻り値の型はint
}
\end{lstlisting}

関数\lstinline!a!へのポインターを使うには関数\lstinline!a!の型が決定していなければならないが、\lstinline!return!文の前に型は決定できないので関数\lstinline!a!はエラーになる。関数\lstinline!b!は\lstinline!return!文が現れた後なので戻り値の型が決定できる。

再帰関数も書ける。

\begin{lstlisting}[language=C++]
auto sum( unsigned int i )
{
    if ( i == 0 )
        return i ; // 戻り値の型はunsigned int
    else
        return sum(i-1)+i ; // OK
}
\end{lstlisting}

このコードも、\lstinline!return!文の順番を逆にすると戻り値の型が決定できずエラーとなるので注意。

\begin{lstlisting}[language=C++]
auto sum( unsigned int i )
{
    if ( i != 0 )
        return sum(i-1)+i ; // エラー
    else
        return i ;
}
\end{lstlisting}

機能テストマクロは~\lstinline!__cpp_return_type_deduction!, 値は201304。
\index{\_\_cpp\_return\_type\_deduction@\texttt{\_\_cpp\_return\_type\_deduction}}
